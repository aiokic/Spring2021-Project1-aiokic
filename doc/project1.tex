% Options for packages loaded elsewhere
\PassOptionsToPackage{unicode}{hyperref}
\PassOptionsToPackage{hyphens}{url}
%
\documentclass[
]{article}
\usepackage{lmodern}
\usepackage{amssymb,amsmath}
\usepackage{ifxetex,ifluatex}
\ifnum 0\ifxetex 1\fi\ifluatex 1\fi=0 % if pdftex
  \usepackage[T1]{fontenc}
  \usepackage[utf8]{inputenc}
  \usepackage{textcomp} % provide euro and other symbols
\else % if luatex or xetex
  \usepackage{unicode-math}
  \defaultfontfeatures{Scale=MatchLowercase}
  \defaultfontfeatures[\rmfamily]{Ligatures=TeX,Scale=1}
\fi
% Use upquote if available, for straight quotes in verbatim environments
\IfFileExists{upquote.sty}{\usepackage{upquote}}{}
\IfFileExists{microtype.sty}{% use microtype if available
  \usepackage[]{microtype}
  \UseMicrotypeSet[protrusion]{basicmath} % disable protrusion for tt fonts
}{}
\makeatletter
\@ifundefined{KOMAClassName}{% if non-KOMA class
  \IfFileExists{parskip.sty}{%
    \usepackage{parskip}
  }{% else
    \setlength{\parindent}{0pt}
    \setlength{\parskip}{6pt plus 2pt minus 1pt}}
}{% if KOMA class
  \KOMAoptions{parskip=half}}
\makeatother
\usepackage{xcolor}
\IfFileExists{xurl.sty}{\usepackage{xurl}}{} % add URL line breaks if available
\IfFileExists{bookmark.sty}{\usepackage{bookmark}}{\usepackage{hyperref}}
\hypersetup{
  pdftitle={Does Trump defeated by covid?},
  pdfauthor={Haosheng Ai},
  hidelinks,
  pdfcreator={LaTeX via pandoc}}
\urlstyle{same} % disable monospaced font for URLs
\usepackage[margin=1in]{geometry}
\usepackage{graphicx,grffile}
\makeatletter
\def\maxwidth{\ifdim\Gin@nat@width>\linewidth\linewidth\else\Gin@nat@width\fi}
\def\maxheight{\ifdim\Gin@nat@height>\textheight\textheight\else\Gin@nat@height\fi}
\makeatother
% Scale images if necessary, so that they will not overflow the page
% margins by default, and it is still possible to overwrite the defaults
% using explicit options in \includegraphics[width, height, ...]{}
\setkeys{Gin}{width=\maxwidth,height=\maxheight,keepaspectratio}
% Set default figure placement to htbp
\makeatletter
\def\fps@figure{htbp}
\makeatother
\setlength{\emergencystretch}{3em} % prevent overfull lines
\providecommand{\tightlist}{%
  \setlength{\itemsep}{0pt}\setlength{\parskip}{0pt}}
\setcounter{secnumdepth}{-\maxdimen} % remove section numbering

\title{Does Trump defeated by covid?}
\usepackage{etoolbox}
\makeatletter
\providecommand{\subtitle}[1]{% add subtitle to \maketitle
  \apptocmd{\@title}{\par {\large #1 \par}}{}{}
}
\makeatother
\subtitle{5243 Project 1}
\author{Haosheng Ai}
\date{1/22/2021}

\begin{document}
\maketitle

\hypertarget{introduction}{%
\section{Introduction}\label{introduction}}

The year 2020 is an unprecedented year that coronavirus changed the
world. While president election caught the eyes all over the world,
coronavirus also played an important role during election. Some media
believe that Trump's terrible performance in handling coronavirus led
him to the lost. In order to confirm that, i conduct an analysis on the
2020 pre-election data.

\hypertarget{about-the-data}{%
\section{About the data}\label{about-the-data}}

The data is the survey responses of voters in the U.S. collected by the
American National Election Studies (ANES). I choose 2020 Exploratory
Testing Survey which includes not only the regular question about
respondents' attitude on candidates but also their views' on covid-19.
Noticed that most of the data was collected through internet instead of
face-to-face interview and most of the surveys were happened near mid of
april.

Since there's no direct data showing the relation between covid-19 and
lost of election. My project focus on several indirect variables that
related to either election or covid-19. The variables i chose can be
categorized by four parts: Birth, Covid-19, Trump's performance , and
who they vote for.

\hypertarget{analysis-visualization}{%
\section{Analysis \& Visualization}\label{analysis-visualization}}

To make the process clear, i conduct my analysis by solving three step
forward question. Question 1: How's Trump performance in handling with
covid? Question 2: How's other factors? Question 3: How people treat
Covid refering to supporting rate?

\hypertarget{question-1-hows-trump-performance-in-handling-with-covid}{%
\subsection{Question 1: How's Trump performance in handling with
covid?}\label{question-1-hows-trump-performance-in-handling-with-covid}}

First I'm curious how respondents view Trumps performance in handling
with Covid-19 as well as Trump's support rate before election. Since age
could make a differ on opinions, so i also conclude age as a variable.
The used survey question is showed below.

\includegraphics{project1_files/figure-latex/data visualization for p1_1-1.pdf}
From the plot, without considering degree of opinion, we could see that
the rates are fair among people. However, the proportion of people who
extremely strongly disapprove Trump's performance is larger than who
extremely strongly approve. Upon this, the public opinion is negative on
this problem.
\includegraphics{project1_files/figure-latex/data visualization for p1_2-1.pdf}
This is a typical question before election to investage each candidate's
support rate. From the graph, Trump is a little behind Biden in all
three groups. Noticed that elder people have higher voting rate to
either Trump or Biden than adults, while they had higher rate of
disapprovement on Trump. This could be a inducement of Trump's lose.
Overall, people had bad impression on Trump's actions on Covid-19 while
Trump had a lower supporting rate. However, the lower supporting rate
may contributed by different reasons,probably not mainly from Covid-19.
So this lead to Question 2.

\hypertarget{question-2-hows-other-factors}{%
\subsection{Question 2: How's other
factors?}\label{question-2-hows-other-factors}}

The survey question on five specific aspects of how Trump did, included
Covid-19, economic, global fairs, healthcare and immigration.

\includegraphics{project1_files/figure-latex/data visualization for p2-1.pdf}
Respondents approve Trump's performance on economic, while disagree on
dealing with healthcare and Covid-19. Refer to Trump's supporting rate,
i do believe that Covid-19 had a large effect on his support rate. In
order to make this conclusion more solid, i further check the relation
between supporting rate of two candidates and people's concerns about
Covids.

\hypertarget{question-3-how-people-treat-covid-refering-to-supporting-rate}{%
\subsection{Question 3: How people treat Covid refering to supporting
rate?}\label{question-3-how-people-treat-covid-refering-to-supporting-rate}}

\includegraphics{project1_files/figure-latex/data visualization for p3-1.pdf}
\includegraphics{project1_files/figure-latex/data visualization for p3-2.pdf}
In general,the graphs show that most respondents are worried of getting
Covid-19 and the economic impact of it. Noticed that more percentage of
people, who have higher degree of concern on Covid-19, willing to vote
for Biden. This analysis result, which reflect the components of
supporting rates, means that more people expected on Biden to deal with
Coronavirus than Trump.

\hypertarget{limitation}{%
\section{Limitation}\label{limitation}}

\begin{enumerate}
\def\labelenumi{\arabic{enumi}.}
\tightlist
\item
  Testing way
\end{enumerate}

Due to coronavirus, almost all of the respondents finish the survey
through the web instead of face-to-face interview. Due to time
consuming, they might not treat the survey seriously thus lead to the
basis. 2. Date of the surveys/n

Since most of the data is collected at the middle of april while
Coronavirus cases just reached its peak. Respondents might not foresee
the post impact from such a large number of cases which may lead to a
basis. Moreover, the election debate, which might largely flip people's
opinions, was in the October. The data would be more reliable if survey
dates are close to the election date.

\hypertarget{conclusion}{%
\section{Conclusion}\label{conclusion}}

In short, Covid-19 did a negative effect on Trump's supporting rate and
is a part of reason on Trump's lose. In this project, I reach the
conclusion by checking people's opinions on Covid-19 as well as other
aspects. Then check the effect of Covid-19 by compared with other
aspects. However, Voting decision is composed by many aspects that's
hard to predict. The result could be more significant with more data
that collecting close to election and including more specific question
on Covid-19.

\end{document}
